%% Classe du document
\documentclass[a4paper,10pt]{article}

%% Francisation
\usepackage[francais]{babel} % Indique que l'on utilise le francais
\usepackage[T1]{fontenc}
\usepackage[utf8]{inputenc} % Indique que l'on utilise tout le clavier
%\usepackage[latin1]{inputenc}

%% Réglages généraux
\usepackage[top=3cm, bottom=3cm, left=3cm, right=3cm]{geometry} % Taille de la feuille
\usepackage{lastpage}

%% Package pour le texte
\usepackage{soul} % Souligner
\usepackage{color} % Utilisation de couleurs
\usepackage{hyperref} % Créer des liens et des signets
\usepackage{eurosym}% Pour le symbole euro
\usepackage{fancyhdr}% Entête et pied de page

%% Package pour les tableaux
\usepackage{multirow} % Colonnes multiples
\usepackage{cellspace}
\usepackage{array}

%% Package pour les dessins
\usepackage{pstricks}
\usepackage{graphicx} % Importer des images
\usepackage{pdftricks} % Pour utiliser avec pdfTex
\usepackage{pst-pdf} % Pour utiliser avec pdfTex
\usepackage{pst-node} % Pose de noeuds
\usepackage{subfig}
\usepackage{float}

%% Package pour les maths
\usepackage{amsmath} % Commandes essentielles
\usepackage{amssymb} % Principaux symboles

%% Package pour le code
\usepackage{listings} % Utilisation de la couleur syntaxique des langages
\usepackage{url}


\usepackage[babel=true]{csquotes} % Permet les quotations (guillemets)
\usepackage{tocvsec2}
\usepackage{amsthm}
\usepackage{amsfonts}

\usepackage{tikz}
\usepackage{pdfpages}

\usetikzlibrary{shapes} % A revoir

%--------------------- Autres définitions ---------------------%

% Propriété des liens
\hypersetup{
colorlinks = true, % Colorise les liens
urlcolor = blue, % Couleur des hyperliens
linkcolor = black, % Couleur des liens internes
}

\definecolor{grey}{rgb}{0.95,0.95,0.95}

% Language Definitions for Turtle
%TODO: a revoir avec les couleur de gedit
\definecolor{olivegreen}{rgb}{0.2,0.8,0.5}
\definecolor{grey2}{rgb}{0.5,0.5,0.5}
\lstdefinelanguage{ttl}{
sensitive=true,
morecomment=[s][\color{grey2}]{@}{:},
morecomment=[l][\color{olivegreen}]{\#},
morecomment=[s][\color{red}]{<}{/>},
morestring=[s][\color{olivegreen}]{<http://w}{\#>},
morestring=[b][\color{blue}]{\"},
}

\lstset{
frame=single,
breaklines=true,
basicstyle=\ttfamily,
backgroundcolor=\color{grey},
basicstyle=\scriptsize,
keywordstyle=\color{blue},
commentstyle=\color{green},
stringstyle=\color{red},
identifierstyle=\color{blue}
}

%Definition de la commande pour retirer l'espace devant les ':'
\makeatletter
\@ifpackageloaded{babel}%
        {\newcommand{\nospace}[1]{{\NoAutoSpaceBeforeFDP{}#1}}}%  % !! double {{}} pour cantonner l'effet à l'argument #1 !!
        {\newcommand{\nospace}[1]{#1}}
\makeatother

\setcounter{tocdepth}{3}
%\maxsecnumdepth{subsubsection} % Dernière section numérotée

\newcommand{\paperPrototyping}{\emph{paper prototyping}}

% Corps du document :
\begin{document}

% Définition des entêtes et pieds de page
\fancyhead[LE,CE,RE,LO,CO,RO]{}
\fancyfoot[LE,CE,RE,LO,CO,RO]{}
\fancyhead[LO, LE]{English}
\fancyhead[RO,RE]{2012/2013}
\fancyfoot[LO,LE]{Université de \scshape{Nantes}}
\fancyfoot[RO,RE]{Page \thepage \ sur \pageref{LastPage}}
\renewcommand{\headrulewidth}{0.4pt}
\renewcommand{\footrulewidth}{0.4pt}

%\maketitle
\begin{titlepage}

\vspace*{\fill}~
\begin{center}
{\large \textsc{Guide}} \\
\vspace{1cm}
{\LARGE How to create a beautiful blog} \\
\vspace{1cm}
COUTABLE Guillaume, MENORET Clément, RULLIER Noémie, WOLLENBURGER Antoine \\
\today
\end{center}
\vspace*{\fill}

\vspace{\stretch{1}}
\begin{center}
\noindent 
\includegraphics[height=2.5cm]{Images/universite.png}
\end{center}
\pagebreak
\end{titlepage}

\newpage
\tableofcontents 

% Introduction
\newpage
\pagestyle{fancy}

%TODO : Correction des fautes d'orthographes, tournures de phrases ...
%TODO : En fonction des parties de tous le monde redire à Nomyx de diminuer ses images pour pas que ça dépasse 10 page (environ)
%TODO : Je ne sais pas si c'est address email ou email address --> J'ai utiliser les deux sens donc à revoir

%%%%%%%%%%%%%%%%%%%%%%%%%%%%%%%%%%%%%%%%%%%%%%%%%%%%%%%%%%%%%%%%%%%%%%%%%%%%%
%%%%%%%%%%  Introduction générale
%%%%%%%%%%%%%%%%%%%%%%%%%%%%%%%%%%%%%%%%%%%%%%%%%%%%%%%%%%%%%%%%%%%%%%%%%%%%%
\section{Introduction}
There are a lots of tools available to create a blog. There are intended to different persons from novice to professional.

This document is a tutorial to help you to create your own blog. We consider that you are not a expert, that's why we choose a tool adapt to all public. We will work with \emph{Overblog}.

To realise this tutorial, you need a internet connection and a valide email address.

%%%%%%%%%%%%%%%%%%%%%%%%%%%%%%%%%%%%%%%%%%%%%%%%%%%%%%%%%%%%%%%%%%%%%%%%%%%%%
%%%%%%%%%%  Etape 1
%%%%%%%%%%%%%%%%%%%%%%%%%%%%%%%%%%%%%%%%%%%%%%%%%%%%%%%%%%%%%%%%%%%%%%%%%%%%%
\newpage
\section{Sign up on overblog}
The first step is to sign up on overblog. I will explain you how to do this:
\begin{enumerate}
\item First, open your web browser (for example \emph{Internet Explorer} \includegraphics[width=0.5cm]{Images/explorer.png} or \emph{Google chrome} \includegraphics[width=0.5cm]{Images/chrome.png}).
\item Go on \emph{Google}, type \emph{overblog} in the search bar and validate. Then click on this link:
\begin{figure}[H]
    \center
	\includegraphics[width=13cm]{Images/linkOverblog.png}
    \caption{Overblog's link}
\end{figure}
Or you can also type in your adress bar this link \url{http://en.over-blog.com/}
\item Now, you have to sign up. Go at the end of the page and click on \emph{Sign up now} like this:
\begin{figure}[H]
    \center
	\includegraphics[width=13cm]{Images/signUpButton.png}
    \caption{Sign up button}
\end{figure}
\item Then you are redirected to this page:
\begin{figure}[H]
    \center
	\includegraphics[width=13cm]{Images/signUpPage.png}
    \caption{Sign up page}
\end{figure}
You have to enter a valide email address because in the next step you have to confirm your registration by clicking a link wich is send on it. Moreover, you have to enter a password and the begining of your address you want for your blog (you have no choice for the end of the address, she must finish with \emph{.over-blog.com}. Here is an example:
\begin{figure}[H]
    \center
	\includegraphics[width=13cm]{Images/signUpPageValues.png}
    \caption{Sign up page with values}
\end{figure}
Then click on \emph{SIGN UP}.
\item Then you are redirected on this page:
\begin{figure}[H]
    \center
	\includegraphics[width=13cm]{Images/addressMailNotConfirmed.png}
    \caption{Email address not confirmed}
\end{figure}
You can see that you have no confirmed you address email. Open your personal information manager, and open the mail send by \emph{marie@overblog.com}. If you don't see this mail, refresh your mail or look in your spam folder. The mail should look like that:
\begin{figure}[H]
    \center
	\includegraphics[width=10cm]{Images/emailOverblog.png}
    \caption{Email send by overblog}
\end{figure}
Click on the link surrounded in red. You are redirected on this page:
\begin{figure}[H]
    \center
	\includegraphics[width=13cm]{Images/addressMailConfirmed.png}
    \caption{Email address confirmed}
\end{figure}
To finish you are redirected on this page:
\begin{figure}[H]
    \center
	\includegraphics[width=13cm]{Images/overblogPage.png}
    \caption{Overblog's home page}
\end{figure}
Now you can start to enrich your blog !
\end{enumerate}


%%%%%%%%%%%%%%%%%%%%%%%%%%%%%%%%%%%%%%%%%%%%%%%%%%%%%%%%%%%%%%%%%%%%%%%%%%%%%
%%%%%%%%%%  Etape 2
%%%%%%%%%%%%%%%%%%%%%%%%%%%%%%%%%%%%%%%%%%%%%%%%%%%%%%%%%%%%%%%%%%%%%%%%%%%%%
\newpage
\section{Create an article}
%TODO Clément
%I - Texte
%II – Media (image/video/son)



%%%%%%%%%%%%%%%%%%%%%%%%%%%%%%%%%%%%%%%%%%%%%%%%%%%%%%%%%%%%%%%%%%%%%%%%%%%%%
%%%%%%%%%%  Etape 3
%%%%%%%%%%%%%%%%%%%%%%%%%%%%%%%%%%%%%%%%%%%%%%%%%%%%%%%%%%%%%%%%%%%%%%%%%%%%%
\newpage
\section{Pimp your blog}
%TODO Guillaume
% Fond d’écran / catégorie / résumé
%personnalisation de base
\subsection{}
With Overblog it is very simple to customize your blog. On the header, you can click on the theme button to access to theme customization page.
\begin{figure}[htpb]
 \centering
 \includegraphics[scale=0.43]{Images/HeaderBar.png}
 \caption{Part of the header}
 \label{customHeader}
\end{figure}

When your are on the theme customization page, on the left panel you can see all the differents theme you can apply on your blog. There is something for
everyone... So when you select a theme you can see the preview, and to apply the theme you should click on the save button, or click on the cancel button if
you want to keep the previous theme.
\begin{figure}[htpb]
 \centering
 \includegraphics[scale=0.43]{Images/ChooseYourTheme.png}
 \caption{The theme chooser}
 \label{customHeader}
\end{figure}
%Personnaliser son blog sur overBlog est vraiment très facile. Sur la barre horizontal supérieur il y a un bouton thème (image du bouton)
%Sur le panneau sur la gauche de votre browser une liste de thème vous est proposé. il y en a pour tout les goûts...
%Le thème est appliqué sur la page d'aperçu pour permettre à M... de voir se que donnera le thème. Lorsque vous avez choisi le thème, il vous suffira de
% cliquer sur le bouton enregistrer(image du bouton) ou le bouton annuler (image du bouton) si au contraire M... souhaite garder le précédent thème.

%personnalisation avancée
\subsection{Advanced customization}
Overblog also provides a configuration for themes. these are not detailed in this document because there are many options. Moreover, these
options can be differents from one theme to another. It is also possible to modify the page html code directly and see the preview. It is discouraged to use
this mode except if you are a confirmed user.
% Overblog propose aussi un mode de configuration pour les thèmes. Je ne les détaillerais pas dans ce document, étant donné qu'il y a de très nombreuses
% options et que ces options peuvent être différentes d'un thème à l'autre. Il est même possible d'avoir et de modifier le contenu de la page html en direct et
% d'en avoir un aperçu. Je vous déconseille d'utiliser le mode html sauf si vous êtes un utilisateur confirmé.


%%%%%%%%%%%%%%%%%%%%%%%%%%%%%%%%%%%%%%%%%%%%%%%%%%%%%%%%%%%%%%%%%%%%%%%%%%%%%
%%%%%%%%%%  Etape 4
%%%%%%%%%%%%%%%%%%%%%%%%%%%%%%%%%%%%%%%%%%%%%%%%%%%%%%%%%%%%%%%%%%%%%%%%%%%%%
\newpage
\section{Sharing it}
%TODO Antoine
% Lien vers extérieur et partage réseaux sociaux (Antoine)

\subsection{Social linkage}

The first time you come to the  \emph{Overblog admin interface} you cannot miss this Panel.
\begin{figure}[h]
    \center
  \includegraphics[width=0.3\textwidth]{Images/shareIni.png}
    \caption{Share panel}
\end{figure}

You can use it by clicking one of the buttons, like the Facebook button (if you have a Facebook account). Then agree with their proposal. With this linkage, you can feed your blog ! You are not obliged to rewrite your Facebook publications.

In another way, this linkage can allow your blog to ask your ``friends" to see your blog.

\subsection{Publish it on your favorite social networks !}

You can like your publications, or quote it by using the bar just under it.

\begin{figure}[h]
    \center
  \includegraphics[width=0.9\textwidth]{Images/articleBar.png}
    \caption{Like bar}
\end{figure}

You are also able to share your article, even your blog, on your favorite social network. You just need to click on one of the following buttons.

\begin{figure}[h]
    \center
  \includegraphics[width=0.5\textwidth]{Images/blogBar.png}
    \caption{Share bar}
\end{figure}

\subsection{Purpose}

But why do all that ? To make your friends come to see your blog, comment it, like it, and share it with their own friends ! So more and more people can come to your blog and see your kiwi jam ! So let's do it.

 


%%%%%%%%%%%%%%%%%%%%%%%%%%%%%%%%%%%%%%%%%%%%%%%%%%%%%%%%%%%%%%%%%%%%%%%%%%%%%
%%%%%%%%%%  CONCLUSION GENERALE
%%%%%%%%%%%%%%%%%%%%%%%%%%%%%%%%%%%%%%%%%%%%%%%%%%%%%%%%%%%%%%%%%%%%%%%%%%%%%
\newpage
\section{Conclusion}
%TODO

\end{document}

\begin{figure}[H]
    \center
    \subfloat[Boîte de dialogue pour le nom de la tâche]{\includegraphics[width=3.7cm]{Images/addTask.png}}\quad
    \subfloat[Tâche]{\includegraphics[width=3.7cm]{Images/listTask.png}}
    \caption{Ajout d'une tâche}
\end{figure}
