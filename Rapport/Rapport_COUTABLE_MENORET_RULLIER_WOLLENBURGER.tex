%% Classe du document
\documentclass[a4paper,10pt]{article}

%% Francisation
\usepackage[francais]{babel} % Indique que l'on utilise le francais
\usepackage[T1]{fontenc}
\usepackage[utf8]{inputenc} % Indique que l'on utilise tout le clavier
%\usepackage[latin1]{inputenc}

%% Réglages généraux
\usepackage[top=3cm, bottom=3cm, left=3cm, right=3cm]{geometry} % Taille de la feuille
\usepackage{lastpage}

%% Package pour le texte
\usepackage{soul} % Souligner
\usepackage{color} % Utilisation de couleurs
\usepackage{hyperref} % Créer des liens et des signets
\usepackage{eurosym}% Pour le symbole euro
\usepackage{fancyhdr}% Entête et pied de page

%% Package pour les tableaux
\usepackage{multirow} % Colonnes multiples
\usepackage{cellspace}
\usepackage{array}

%% Package pour les dessins
\usepackage{pstricks}
\usepackage{graphicx} % Importer des images
\usepackage{pdftricks} % Pour utiliser avec pdfTex
\usepackage{pst-pdf} % Pour utiliser avec pdfTex
\usepackage{pst-node} % Pose de noeuds
\usepackage{subfig}
\usepackage{float}

%% Package pour les maths
\usepackage{amsmath} % Commandes essentielles
\usepackage{amssymb} % Principaux symboles

%% Package pour le code
\usepackage{listings} % Utilisation de la couleur syntaxique des langages
\usepackage{url}


\usepackage[babel=true]{csquotes} % Permet les quotations (guillemets)
\usepackage{tocvsec2}
\usepackage{amsthm}
\usepackage{amsfonts}

\usepackage{tikz}
\usepackage{pdfpages}

\usetikzlibrary{shapes} % A revoir

%--------------------- Autres définitions ---------------------%

% Propriété des liens
\hypersetup{
colorlinks = true, % Colorise les liens
urlcolor = blue, % Couleur des hyperliens
linkcolor = black, % Couleur des liens internes
}

\definecolor{grey}{rgb}{0.95,0.95,0.95}

% Language Definitions for Turtle
%TODO: a revoir avec les couleur de gedit
\definecolor{olivegreen}{rgb}{0.2,0.8,0.5}
\definecolor{grey2}{rgb}{0.5,0.5,0.5}
\lstdefinelanguage{ttl}{
sensitive=true,
morecomment=[s][\color{grey2}]{@}{:},
morecomment=[l][\color{olivegreen}]{\#},
morecomment=[s][\color{red}]{<}{/>},
morestring=[s][\color{olivegreen}]{<http://w}{\#>},
morestring=[b][\color{blue}]{\"},
}

\lstset{
frame=single,
breaklines=true,
basicstyle=\ttfamily,
backgroundcolor=\color{grey},
basicstyle=\scriptsize,
keywordstyle=\color{blue},
commentstyle=\color{green},
stringstyle=\color{red},
identifierstyle=\color{blue}
}

%Definition de la commande pour retirer l'espace devant les ':'
\makeatletter
\@ifpackageloaded{babel}%
        {\newcommand{\nospace}[1]{{\NoAutoSpaceBeforeFDP{}#1}}}%  % !! double {{}} pour cantonner l'effet à l'argument #1 !!
        {\newcommand{\nospace}[1]{#1}}
\makeatother

\setcounter{tocdepth}{3}
%\maxsecnumdepth{subsubsection} % Dernière section numérotée

\newcommand{\paperPrototyping}{\emph{paper prototyping}}

% Corps du document :
\begin{document}

% Définition des entêtes et pieds de page
\fancyhead[LE,CE,RE,LO,CO,RO]{}
\fancyfoot[LE,CE,RE,LO,CO,RO]{}
\fancyhead[LO, LE]{English}
\fancyhead[RO,RE]{2012/2013}
\fancyfoot[LO,LE]{Université de \scshape{Nantes}}
\fancyfoot[RO,RE]{Page \thepage \ sur \pageref{LastPage}}
\renewcommand{\headrulewidth}{0.4pt}
\renewcommand{\footrulewidth}{0.4pt}

%\maketitle
\begin{titlepage}

\vspace*{\fill}~
\begin{center}
{\large \textsc{Guide}} \\
\vspace{1cm}
{\LARGE How to create a beautiful blog} \\
\vspace{1cm}
COUTABLLE Guillaume, MENORET Clément, RULLIER Noémie, WOLLENBURGER Antoine \\
\today
\end{center}
\vspace*{\fill}

\vspace{\stretch{1}}
\begin{center}
\noindent 
\includegraphics[height=2.5cm]{Images/universite.png}
\end{center}
\pagebreak
\end{titlepage}

\newpage
\tableofcontents 

% Introduction
\newpage
\pagestyle{fancy}

%%%%%%%%%%%%%%%%%%%%%%%%%%%%%%%%%%%%%%%%%%%%%%%%%%%%%%%%%%%%%%%%%%%%%%%%%%%%%
%%%%%%%%%%  Introduction générale
%%%%%%%%%%%%%%%%%%%%%%%%%%%%%%%%%%%%%%%%%%%%%%%%%%%%%%%%%%%%%%%%%%%%%%%%%%%%%
\section{Introduction}
%TODO Noémie

%%%%%%%%%%%%%%%%%%%%%%%%%%%%%%%%%%%%%%%%%%%%%%%%%%%%%%%%%%%%%%%%%%%%%%%%%%%%%
%%%%%%%%%%  Etape 1
%%%%%%%%%%%%%%%%%%%%%%%%%%%%%%%%%%%%%%%%%%%%%%%%%%%%%%%%%%%%%%%%%%%%%%%%%%%%%
\newpage
\section{Create and sign up }
%TODO Noémie



%%%%%%%%%%%%%%%%%%%%%%%%%%%%%%%%%%%%%%%%%%%%%%%%%%%%%%%%%%%%%%%%%%%%%%%%%%%%%
%%%%%%%%%%  Etape 2
%%%%%%%%%%%%%%%%%%%%%%%%%%%%%%%%%%%%%%%%%%%%%%%%%%%%%%%%%%%%%%%%%%%%%%%%%%%%%
\newpage
\section{Create an article}
%TODO Clément
%I - Texte
%II – Media (image/video/son)



%%%%%%%%%%%%%%%%%%%%%%%%%%%%%%%%%%%%%%%%%%%%%%%%%%%%%%%%%%%%%%%%%%%%%%%%%%%%%
%%%%%%%%%%  Etape 3
%%%%%%%%%%%%%%%%%%%%%%%%%%%%%%%%%%%%%%%%%%%%%%%%%%%%%%%%%%%%%%%%%%%%%%%%%%%%%
\newpage
\section{Customize your blog}
%TODO Guillaume
% Fond d’écran / catégorie / résumé
%personnalisation de base
%Personnaliser son blog sur overBlog est vraiment très facile. Sur la barre horizontal supérieur il y a un bouton thème (image du bouton)
%Sur le panneau sur la gauche de votre browser une liste de thème vous est proposé. il y en a pour tout les goûts...
%Le thème est appliqué sur la page d'aperçu pour permettre à M... de voir se que donnera le thème. Lorsque vous avez choisi le thème, il vous suffira de
% cliquer sur le bouton enregistrer(image du bouton) ou le bouton annuler (image du bouton) si au contraire M... souhaite garder le précédent thème.

%personnalisation avancée
% Overblog propose aussi un mode de configuration pour les thèmes. Je ne les détaillerais pas dans ce document, étant donné qu'il y a de très nombreuses
% options et que ces options peuvent être différentes d'un thème à l'autre. Il est même possible d'avoir et de modifier le contenu de la page html en direct et
% d'en avoir un aperçu. Je vous déconseille d'utiliser le mode html sauf si vous êtes un utilisateur confirmé.


%%%%%%%%%%%%%%%%%%%%%%%%%%%%%%%%%%%%%%%%%%%%%%%%%%%%%%%%%%%%%%%%%%%%%%%%%%%%%
%%%%%%%%%%  Etape 4
%%%%%%%%%%%%%%%%%%%%%%%%%%%%%%%%%%%%%%%%%%%%%%%%%%%%%%%%%%%%%%%%%%%%%%%%%%%%%
\newpage
\section{Share}
%TODO Antoine
% Fond d’écran / catégorie / résumé
% Lien vers extérieur et partage réseaux sociaux (Antoine)


%%%%%%%%%%%%%%%%%%%%%%%%%%%%%%%%%%%%%%%%%%%%%%%%%%%%%%%%%%%%%%%%%%%%%%%%%%%%%
%%%%%%%%%%  CONCLUSION GENERALE
%%%%%%%%%%%%%%%%%%%%%%%%%%%%%%%%%%%%%%%%%%%%%%%%%%%%%%%%%%%%%%%%%%%%%%%%%%%%%
\newpage
\section{Conclusion}
%TODO

\end{document}


